\documentclass{article}
\usepackage{multicol}
\usepackage[a4paper, margin=1.5cm]{geometry}
\usepackage{listings}
\usepackage[spanish]{babel}
\usepackage{enumitem}


\title{Trabajo Práctico 2}
\author{Federico Pizarro, Agustín Pereyra}
\date{\today}
\begin{document}



\maketitle
\begin{multicols}{2}

\section{Pokedex}


\section{Control de Aeronave en Hangar Automatizado}


\section{Sistema de Monitoreo y Procesamiento de Robots Autónomos}
\quad Para la resolución de esta parte, se hizo uso de una sola estructura
llamada \lstinline|Task|, la cuál tiene las propiedades de las tareas
emitidas por los Sensores y procesadas por los Robots, ambos
implementados como hilos (threads). Además está casi todo el código
dentro del mismo archivo source, main.cpp, y lo único fuera de él es tanto
la implementación como la definición de la estructura.

\quad El struct \lstinline|Task| solo contiene el ID del mismo, el ID del
sensor que lo generó y una descripción del mismo, la cual se elige aleatoriamente
de un array de descripciones genéricas. Además tiene solo un constructor que tiene
como función adicional, simular el delay de 175ms de la creación de una tarea.

\quad Cuando el programa comienza, se definen tanto las constantes como las variables
globales, entre ellas podemos destacar a los 2 objetos \lstinline|std::mutex|, encargados
de sincronizar tanto el uso de la consola como de la \lstinline|std::queue| de tasks,
al objeto \lstinline|std::atomic| que lleva la cuenta de los sensores activos y a la variable
y a la variable condicional \lstinline|std::condition\_variable| usada para sincronizar a
los sensores con los robots.

Luego se inician los threads definidos como \lstinline|std::jthreads| para que todos sean
ejecutados por completo y no puedan ser interrumpidos. Luego el funcionamiento se rige por 
la siguiente estructura para cada hilo:

\subsection{Sensores}
\begin{enumerate}[label=\Roman*.]
    \item Por cada tarea a generar 
    \begin{enumerate}[label=\roman*.]
        \item Se bloquea el \lstinline|std::mutex| de la queue para crear la tarea,
        asignando los atributos y agregándola a la cola.
        \item Se bloque la consola y se hace el \textit{log} respectivo.
        \item Se notifica a los robots (que están en espera) que hay una tarea disponible
        mediante la variable condicional.
    \end{enumerate}
    \item Lockeando el mutex destinado a los sensores, se actualiza el contador atómico y se 
    verifica si es el último sensor. Si es el caso se notifica a los robots.
\end{enumerate}

\subsection{Robots}
\begin{enumerate}[label=\Roman*.]
    \item Mientras que hayan sensores activos o tareas disponibles
    \begin{enumerate}[label=\roman*.]
        \item Espera hasta ser notificado por los sensores que hay una tarea nueva disponible,
        o que terminó el último sensor
        \item Se quita la tarea de la cola y se la procesa (esperando el respectivo acceso mediante
        el mutex)
        \item Se simula el delay del procesamiento de la tarea de 250ms
        \item Se hace el \textit{log} del procesamiento lockeando primero la consola.
    \end{enumerate}
    \item Se loggea la finalización del trabajo del Robot
\end{enumerate}
\section*{Aprendizaje}
\end{multicols}
\end{document}